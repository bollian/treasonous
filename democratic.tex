\documentclass{report}

\usepackage{constitutional}

\title{Constitution}
\newcommand{\nation}{The Nation}

\begin{document}
    \maketitle
    \section{Preamble}
    The goal of this document is to establish a democratically elected government,
    whose purpose is to provide for the welfare, defense, and justice of \nation.
    This government shall operate under the rule of law and protect the rights of
    the people

    \section{Article 1}

    Parliament is granted all legislative powers of \nation, and shall be made
    up of 125 Representatives.

    Sixty-two Representatives are elected in local districts using instant-runoff
    voting. Sixty-three Representatives are selected by the parties in a
    mixed-member voting system.

    A Parliamentary election is triggered by a simple majority vote in Parliament
    or when 5 years have passed since the previous election. After each election,
    if a single party has a majority in Parliament, it forms the Government. If
    no single party has a majority, a number of parties necessary to possess a
    majority must join together to form a coalition Government. If the Government
    cannot be formed, another election is held.

    Each time an election is held and a Government is formed, the President
    chooses, from the various Representatives belonging to the Government, a
    single person to appoint as the Prime Minister. The Prime Minister may then
    choose Ministers from all the elected Representatives to head the executive
    offices of the state.

    Parliament must assemble for session at least once a year, and a majority of
    Representatives must be present in order to constitute a quorum. If the
    annual budget or issues related to national security, territorial integrity,
    and national sovereignty are under discussion, Parliament may not recess
    until the matter is decided.

    A Representative must be a citizen of \nation{} for at least seven years and
    be at least 25 years old. If a Representative is elected for an electoral
    district, they must be a resident of the district for which they are
    elected. No person may simultaneously be a Representative and the President
    or a Justice of a court.

    Parliament may, by a simple majority vote and pending signature by the
    President, establish taxes and tariffs, declare war, provide for the
    funding of the bureaucracy and salaries for public office holders, and pass
    any laws or acts necessary to fulfill any functions of the state. The
    salaries of public office holders shall not be increased during a term in
    office.

    Parliament may, by a simple majority vote, approve treaties negotiated and
    submitted by the President. After being approved by Parliament, the
    President may sign the treaty into law or veto.

    Parliament may not pass ex post facto legislation nor bills of attainder.

    Parliament may, by a simple majority vote, establish rules governing the
    procedures of Parliament.

    Parliament may, by a simple-majority vote and pending signature by the
    President, repeal previously passed legislation.

    Parliament may, by a two-thirds majority vote and pending signature by
    the President, modify the rules governing the process for naturalizing new
    citizens.

    Parliament may, by a three-fourths majority vote, remove the President or a
    court Justice from office.

    \section{Article 2}

    The executive office is held by a President, who is reponsible for
    representing \nation{} to foreign governments.

    The President has the sole power to pardon those convicted of a criminal
    offense or commute their sentences, and vto or sign legislation passed by
    Parliament. Any legislation that isn't vetoed by the President within 7 days
    is considered signed and approved.

    Pending majority approval by Parliament, the President has the sole power
    to appoint Justices to the Supreme Court and other national courts as well
    as appoint ambassadors.

    The President may summon Parliament to hold an emergency session and present
    matters to discuss to Parliament.

    The President may announce a Parliamentary election no sooner than one year
    after the previous Parliamentary or Presidential election.

    The President is the Commander in Chief of the military, and may exercise
    those powers in times of war or should the sovereignty or territorial
    integrity of \nation{} come under attack.

    The President must be a citizen of the nation for at least 10 years and be
    at least 35 years old. No person may hold the office of the President for
    more than three terms. The President must reject and renounce the titles,
    offices, and citizenships of any foreign states or nations. They must also
    reject any gifts from foreign states or nations that aren't purely
    cermonial in nature. The President may not be a Representative nor a court
    Justice while holding the office of the Presidency.

    Presidential elections are held every five years and begin on the first
    Saturday in the month of December and continue into the following Sunday.
    Presidential elections use instant-runoff voting.

    On the first day of the first year after an election, the President-Elect
    takes the oath of office and is sworn in by the Chief Justice of the
    Supreme Court.

    In the case of a vacancy of the Presidency, the office will be filled by the
    next available person in a line of succession that begins with the Prime
    Minister, then the members of the Cabinet, and then the Representatives in
    Parliament. The order in which Cabinet members and Representatives are
    chosen is done in a manner set by law.

\end{document}
